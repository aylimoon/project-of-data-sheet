% Б.УЗИ B.USD Виртуальный симулятор ультразвуковой диагностики

%---Используемые переменные (справочный раздел)--- % Коммент этого сектора должен быть всегда, этот сектор для справки
% используются в операторах case
%\newcommand{\typesimulator}{0}   %тип симулятора (используется в операторе кейс) значения (0,1,2)=(Б,С,Г)
%\newcommand{\typemanufacturer}{0} % выбор компании

%обычные переменные
%\newcommand{\producttype}{Б.ЛПР} %тип продукта    
%\newcommand{\productname}{Виртуальный симулятор лапароскопии}
%\newcommand{\productmainview}{lapvision-standart} % внешний вид продукта    
%\newcommand{\productsertificate}{ce_sertificate_em} % сертификат продукта    
%\newcommand{\producttrademark}{LapVision} % торговое наименование симулятора
%\newcommand{\producttrademarktype}{Standart} % торговое наименование типа симулятора

%\newcommand{\productserialnumber}{/\_\_\_\_\_\_\_\_\_\_\_\_\_\_\_\_\_\_\_\_\_/}
%\newcommand{\productiondate}{2019}  %дата производства

%\newcommand{\productmadein}{Сделано в} % страна производства   
%\newcommand{\manufacturername}{ООО~«Эйдос-Медицина»} %название компании
%\newcommand{\address}{420107,  Россия, г.~Казань, ул.~Петербургская, д.50}
%\newcommand{\phone}{+7 (843) 227-40-63} 
%\newcommand{\email}{mail@oooeidos.ru}

%\newcommand{\producttechchar} % Технические характеристики симулятора
    %\newcommand{\producttechchardimension} % Габаритные размеры симулятора
    %\newcommand{\producttechcharweighs} % Вес симулятора 
%\newcommand{\productequip} % Комплектация симулятора
%---Используемые переменные (справочный раздел)---


\ifdefined \typesimulator
\else
%{
%---Определение типа симулятора---  что-то одно всегда должно быть включено
\newcommand{\typesimulator}{0}   % Базовая модификация
%\newcommand{\typesimulator}{1}   % Стандартная модификация
%\newcommand{\typesimulator}{2}   % Гибридная модификация
%\newcommand{\typesimulator}{3}     % Ручная модификация, если комплектацию и тип симулятора нужно вбить в ручную 
                                    %(редактировать ниже в операторе ifcase \typesimulator    на 3 позиции)
%---Определение типа симулятора---
%}
\fi


% Определение общих переменных (рус и англ)

\newcommand{\producttrademark}{SonoVision}
\newcommand{\producttrademarknameone}{Sono}
\newcommand{\producttrademarknametwo}{Vision}
\newcommand{\producttrademarknametitul}
{
\large\textbf{\textcolor{TradeMarkNameOneColor}{\producttrademarknameone}\textcolor{TradeMarkNameTwoColor}{\producttrademarknametwo}} %~\textcolor{TradeMarkNameOneColor}{\producttrademarktype}}
}
\newcommand{\producttechcharvoltage}{110 - 220} % Напряжение питания

\ifcase \typesimulator   % кейс для всех языков
%{Базовая модификация} 
\newcommand{\productmainview}{B_USD_2}
\newcommand{\producttrademarktype}{}
%\newcommand{\producttechchardimension}{1300x500x400} % Габаритные размеры симулятора
\ifIsLngRus\newcommand{\producttechchardimension}{1300x500x400}\else\newcommand{\producttechchardimension}{1,300x500x400}\fi % Габаритные размеры симулятора
\newcommand{\producttechcharweighs}{30} % Вес симулятора
\newcommand{\producttechcharpower}{500} % Потребляемая мощность

\or
%{Стандартная модификация}
\newcommand{\productmainview}{image-not-found}
\newcommand{\producttrademarktype}{STANDARD}
\ifIsLngRus\newcommand{\producttechchardimension}{550x1000x1800}\else\newcommand{\producttechchardimension}{550x1,000x1,800}\fi % Габаритные размеры симулятора
%\newcommand{\producttechchardimension}{550x1000x1800} % Габаритные размеры симулятора
\newcommand{\producttechcharweighs}{90} % Вес симулятора
\ifIsLngRus\newcommand{\producttechcharpower}{1000}\else\newcommand{\producttechcharpower}{1,000}\fi % Потребляемая мощность

\or
%{Гибридная модификация} 
\newcommand{\productmainview}{image-not-found}
\newcommand{\producttrademarktype}{HYBRID}
\ifIsLngRus\newcommand{\producttechchardimension}{3000x2800x1800}\else\newcommand{\producttechchardimension}{3,000x2,800x1,800}\fi % Габаритные размеры симулятора
\newcommand{\producttechcharweighs}{170} % Вес симулятора
\ifIsLngRus\newcommand{\producttechcharpower}{2000}\else\newcommand{\producttechcharpower}{2,000}\fi % Потребляемая мощность

\or
%{Базовая модификация}  Демонстрационное оборудование
\newcommand{\productmainview}{image-not-found}
\newcommand{\producttrademarktype}{SMART}
\newcommand{\producttechchardimension}{550x650x650} % Габаритные размеры симулятора
\newcommand{\producttechcharweighs}{30} % Вес симулятора
\newcommand{\producttechcharpower}{500} % Потребляемая мощность

\else
% Ручная модификация
\newcommand{\productmainview}{image-not-found}
\newcommand{\producttrademarktype}{NoName}
\newcommand{\producttechchardimension}{XXXхXXхXXXXX} % Габаритные размеры симулятора
\newcommand{\producttechcharweighs}{XXX} % Вес симулятора
\newcommand{\producttechcharpower}{XXX} % Потребляемая мощность

\fi

%\IsLngRusfalse % Можно использовать при отладке, но не забыть закомментировать перед компиляцией bat файлом
\ifIsLngRus % определение переменных на русском языке

    %---Определение значения переменных не зависящих от типа симулятора---
    \newcommand{\productname}{Виртуальный симулятор ультразвуковой диагностики}

    %---Определение значения переменных не зависящих от типа симулятора---

    %---Определение переменных зависящих от типа симулятора---
    \ifcase \typesimulator
    %{Базовая модификация} 

    \newcommand{\producttype}{Б.УЗИ}

    %---Комплектация---
    \newcommand{\productequip}
    {
        Полуторсовый манекен & \\ \hline
        Имитатор трансторакального  датчика УЗИ & \\ \hline
        Имитатор трансабдоминального датчика УЗИ & \\ \hline
        Имитатор линейного датчика УЗИ & \\ \hline
        Имитатор интравагинального датчика УЗИ & \\ \hline
        Имитатор трансэзофагеального датчика УЗИ & \\ \hline
        Электромагнитный трекер с сенсорами & \\ \hline
        Ноутбук & \\ \hline
        Мышь & \\ \hline
        USB-ключ & \\ \hline
        Сетевой фильтр & \\ \hline
        Кейс & \\ \hline
        Технический паспорт изделия & \\ \hline
        Руководство по эксплуатации & \\ \hline    
              
          
        &\\ \hline
        %&\\ \hline
        %&\\ \hline 
    }

    \or

    %{Стандартная модификация}
    \newcommand{\producttype}{C.УЗИ}

    %---Комплектация---
    \newcommand{\productequip}
    {
        
    
        &\\ \hline
        &\\ \hline
        &\\ \hline    
    }
    \or

    %{Гибридная модификация} 
    \newcommand{\producttype}{Г.УЗИ}

    %---Комплектация---
    \newcommand{\productequip}
    {
        
        
        &\\ \hline
        &\\ \hline
        &\\ \hline
        &\\ \hline
        &\\ \hline 
    }
    
    \or 

    % Ручная модификация
    \newcommand{\producttype}{Б.ХХХ}

    \newcommand{\producttechchar}
    {
        Поле 1  & 1 \\ \hline 
        Поле 2  & 2 \\ \hline %название не очень
        Поле 3  & 3 \\ \hline 
    }
    \newcommand{\productequip}
    {
        Поле 1  & 1 \\ \hline 
        Поле 2  & 2 \\ \hline %название не очень
        Поле 3  & 3 \\ \hline 
        Технический паспорт	& 1 \\ \hline
        Руководство по эксплуатации	& 1 \\ \hline
    }
    \fi
    %---Определение переменных зависящих от типа симулятора---

%%%%%%%%%%%%%%%%%%%%%%%%%%%%%%%%%%%%%%%%%%%%%%%%%%%%%%%%%%%%%%%%%%%%%%%%%%%%%%%%%%%%%%%%%%%%%%%%%%%%%%%%%%
%%%%%%%%%%%%%%%%%%%%%%%%%%%%%%%%%%%%%%%%%%%%%%%%%%%%%%%%%%%%%%%%%%%%%%%%%%%%%%%%%%%%%%%%%%%%%%%%%%%%%%%%%%
%%%%%%%%%%%%%%%%%%%%%%%%%%%%%%%%%%%%%%%%%%%%%%%%%%%%%%%%%%%%%%%%%%%%%%%%%%%%%%%%%%%%%%%%%%%%%%%%%%%%%%%%%%
%%%%%%%%%%%%%%%%%%%%%%%%%%%%%%%%%%%%%%%%%%%%%%%%%%%%%%%%%%%%%%%%%%%%%%%%%%%%%%%%%%%%%%%%%%%%%%%%%%%%%%%%%%
%%%%%%%%%%%%%%%%%%%%%%%%%%%%%%%%%%%%%%%%%%%%%%%%%%%%%%%%%%%%%%%%%%%%%%%%%%%%%%%%%%%%%%%%%%%%%%%%%%%%%%%%%%
%%%%%%%%%%%%%%%%%%%%%%%%%%%%%%%%%%%%%%%%%%%%%%%%%%%%%%%%%%%%%%%%%%%%%%%%%%%%%%%%%%%%%%%%%%%%%%%%%%%%%%%%%%

\else % определение переменных на английском языке

    %---Определение значения переменных не зависящих от типа симулятора---
    \newcommand{\productname}{Virtual Simulator for Practical Skills in Diagnostic Ultrasonography}

    %---Определение значения переменных не зависящих от типа симулятора---

    %---Определение переменных зависящих от типа симулятора---
    \ifcase \typesimulator
    %{Базовая модификация} 

    \newcommand{\producttype}{B.USD}

    %---Комплектация---
    \newcommand{\productequip}
    {
       
        Half-body manikin & \\ \hline
        Transthoracic ultrasonic transducer imitator & \\ \hline
        Transabdominal ultrasonic transducer imitator & \\ \hline
        Linear array ultrasonic transducer imitator & \\ \hline
        Transvaginal ultrasonic transducer imitator & \\ \hline
        Transesophageal ultrasonic transducer imitator & \\ \hline
        Electromagnetic tracker with sensor & \\ \hline
        Laptop & \\ \hline
        Mouse & \\ \hline
        USB key & \\ \hline
        Power strip & \\ \hline
        Transportation case & \\ \hline
        Product Data Sheet & \\ \hline
        User Manual & \\ \hline    
        
        &\\ \hline
        %&\\ \hline
        %&\\ \hline
        %&\\ \hline
        %&\\ \hline 
    }

    \or

    %{Стандартная модификация}
    \newcommand{\producttype}{S.USD}

    %---Комплектация---
    \newcommand{\productequip}
    {
   

          
        &\\ \hline
        &\\ \hline
        &\\ \hline    
    }
    \or

    %{Гибридная модификация} 
    \newcommand{\producttype}{H.USD}

    %---Комплектация---
    \newcommand{\productequip}
    {
        
       
        &\\ \hline
        &\\ \hline
        &\\ \hline
        &\\ \hline
        &\\ \hline 
    }    
  
    
    \or 

    % Ручная модификация
    \newcommand{\producttype}{Б.ХХХ}


    \newcommand{\producttechchar}
    {
        Поле 1  & 1 \\ \hline 
        Поле 2  & 2 \\ \hline %название не очень
        Поле 3  & 3 \\ \hline 
    }
    \newcommand{\productequip}
    {
        Поле 1  & 1 \\ \hline 
        Поле 2  & 2 \\ \hline %название не очень
        Поле 3  & 3 \\ \hline 
        Технический паспорт	& 1 \\ \hline
        Руководство по эксплуатации	& 1 \\ \hline
    }  
    \fi
    %---Определение переменных зависящих от типа симулятора--- 

\fi











