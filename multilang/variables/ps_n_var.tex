% РП.Н PS.N Симулятор пациента новорожденного

%---Используемые переменные (справочный раздел)--- % Коммент этого сектора должен быть всегда, этот сектор для справки
% используются в операторах case
%\newcommand{\typesimulator}{0}   %тип симулятора (используется в операторе кейс) значения (0,1,2)=(Б,С,Г)
%\newcommand{\typemanufacturer}{0} % выбор компании

%обычные переменные
%\newcommand{\producttype}{Б.ЛПР} %тип продукта    
%\newcommand{\productname}{Виртуальный симулятор лапароскопии}
%\newcommand{\productmainview}{lapvision-standart} % внешний вид продукта    
%\newcommand{\productsertificate}{ce_sertificate_em} % сертификат продукта    
%\newcommand{\producttrademark}{LapVision} % торговое наименование симулятора
%\newcommand{\producttrademarktype}{Standart} % торговое наименование типа симулятора

%\newcommand{\productserialnumber}{/\_\_\_\_\_\_\_\_\_\_\_\_\_\_\_\_\_\_\_\_\_/}
%\newcommand{\productiondate}{2019}  %дата производства

%\newcommand{\productmadein}{Сделано в} % страна производства   
%\newcommand{\manufacturername}{ООО~«Эйдос-Медицина»} %название компании
%\newcommand{\address}{420107,  Россия, г.~Казань, ул.~Петербургская, д.50}
%\newcommand{\phone}{+7 (843) 227-40-63} 
%\newcommand{\email}{mail@oooeidos.ru}

%\newcommand{\producttechchar} % Технические характеристики симулятора
    %\newcommand{\producttechchardimension} % Габаритные размеры симулятора
    %\newcommand{\producttechcharweighs} % Вес симулятора 
%\newcommand{\productequip} % Комплектация симулятора
%---Используемые переменные (справочный раздел)---


\ifdefined \typesimulator
\else
%{
%---Определение типа симулятора---  что-то одно всегда должно быть включено
\newcommand{\typesimulator}{0}   % РП.Н.ХФ Симулятор пациента новорожденного
%\newcommand{\typesimulator}{1}   % РП.Н.ЭС Отсутствует
%\newcommand{\typesimulator}{2}   % РП.Н.Б Отсутствует
%\newcommand{\typesimulator}{3}   % РП.Н.СЛР Отсутствует 
%\newcommand{\typesimulator}{4}     % Ручная модификация, если комплектацию и тип симулятора нужно вбить в ручную 
                                    %(редактировать ниже в операторе ifcase \typesimulator    на 3 позиции)
%---Определение типа симулятора---
%}
\fi

%\IsLngRusfalse % Можно использовать при отладке, но не забыть закомментировать перед компиляцией bat файлом

% Определение общих переменных (рус и англ)

\newcommand{\producttrademark}{MIA}
\newcommand{\producttrademarknameone}{MIA}
\newcommand{\producttrademarknametwo}{Vision}
\newcommand{\producttrademarknametitul}
{
\large\textbf{\textcolor{TradeMarkNameOneColor}{\producttrademarknameone}~\textcolor{TradeMarkNameOneColor}{\producttrademarktype}}
}
\newcommand{\producttechcharvoltage}{110 - 220} % Напряжение питания

\ifcase \typesimulator   % кейс для всех языков
% РП.Н.ХФ 
\newcommand{\productmainview}{PS_N-1}
\newcommand{\producttrademarktype}{HF}
\newcommand{\producttechchardimension}{550x220x180} % Габаритные размеры симулятора
\newcommand{\producttechcharweighs}{20} % Вес симулятора
\newcommand{\producttechcharpower}{150} % Потребляемая мощность

\or
% РП.Н.ЭС
\newcommand{\productmainview}{image-not-found}
\newcommand{\producttrademarktype}{Essential}
\ifIsLngRus\newcommand{\producttechchardimension}{1200x420x420}\else\newcommand{\producttechchardimension}{1,200x420x420}\fi % Габаритные размеры симулятора
\newcommand{\producttechcharweighs}{35} % Вес симулятора
\ifIsLngRus\newcommand{\producttechcharpower}{500}\else\newcommand{\producttechcharpower}{1,000}\fi % Потребляемая мощность

\or
% РП.Н.Б
\newcommand{\productmainview}{image-not-found}
\newcommand{\producttrademarktype}{BLS}
\newcommand{\producttechchardimension}{3000x2800x1800} % Габаритные размеры симулятора
\newcommand{\producttechcharweighs}{35} % Вес симулятора
\newcommand{\producttechcharpower}{2000} % Потребляемая мощность

\or
% РП.Н.СЛР
\newcommand{\productmainview}{image-not-found}
\newcommand{\producttrademarktype}{CPR}
\ifIsLngRus\newcommand{\producttechchardimension}{1200x420x420}\else\newcommand{\producttechchardimension}{1,200x420x420}\fi % Габаритные размеры симулятора
\newcommand{\producttechcharweighs}{35} % Вес симулятора
\newcommand{\producttechcharpower}{500} % Потребляемая мощность

\else
% Ручная модификация
\newcommand{\productmainview}{image-not-found}
\newcommand{\producttrademarktype}{NoName}
\newcommand{\producttechchardimension}{XXXхXXхXXXXX} % Габаритные размеры симулятора
\newcommand{\producttechcharweighs}{XXX} % Вес симулятора
\newcommand{\producttechcharpower}{XXX} % Потребляемая мощность

\fi

%\IsLngRusfalse % Можно использовать при отладке, но не забыть закомментировать перед компиляцией bat файлом
\ifIsLngRus % определение переменных на русском языке

    %---Определение значения переменных не зависящих от типа симулятора---
    \newcommand{\productname}{Симулятор пациента новорожденного}

    %---Определение значения переменных не зависящих от типа симулятора---

    %---Определение переменных зависящих от типа симулятора---
    \ifcase \typesimulator
    % РП.Н.ХФ

    \newcommand{\producttype}{РП.Н.ХФ}

    %---Комплектация---
    \newcommand{\productequip}
    {
        Симулятор пациента & \\ \hline
        Зарядное устройство симулятора пациента & \\ \hline
        
        Адаптер & \\ \hline
        Зарядное устройство Адаптера & \\ \hline
        Кабель Адаптера на 4 электрода & \\ \hline
        Кабель-переходник Адаптер - канал ЭКГ & \\ \hline
        Кабель-переходник Адаптер - канал ЭКС & \\ \hline
        Кабель переходник Адаптер - канал SpO2 & \\ \hline
        Накладки дефибриллятора & \\ \hline
        
        Тонометр с фонендоскопом & \\ \hline
        Трубка дренажная & \\ \hline
        
        Расходные материалы для вливания в вену головы & \\ \hline
        Расходные материалы для гемоторакса & \\ \hline        
        
        Расходные материалы для внутрикостного вливания & \\ \hline
        
        Одежда: шапка & \\ \hline
        Ноутбук & \\ \hline
        
        Моноблок & \\ \hline
        
        Wi-Fi роутер & \\ \hline
        
        Технический паспорт изделия & \\ \hline
        
        Руководство по эксплуатации & \\ \hline       

    
        %&\\ \hline
        %&\\ \hline
        %&\\ \hline 
    }

    \or

    % РП.Н.ЭС
    \newcommand{\producttype}{РП.Н.ЭС}

    %---Комплектация---
    \newcommand{\productequip}
    {

    
        
        &\\ \hline
        &\\ \hline
        &\\ \hline
        %&\\ \hline
        %&\\ \hline    
    }
    \or

    % РП.Н.Б 
    \newcommand{\producttype}{РП.Н.Б}

    %---Комплектация---
    \newcommand{\productequip}
    {
        Поле 1  & 1 \\ \hline 
        Поле 2  & 2 \\ \hline %название не очень
        Поле 3  & 3 \\ \hline 

        &\\ \hline
        &\\ \hline
        &\\ \hline
        &\\ \hline
        &\\ \hline 
    }
    \or

    % РП.Н.СЛР
    \newcommand{\producttype}{РП.Н.СЛР}

    %---Комплектация---
    \newcommand{\productequip}
    {
        
        &\\ \hline 
    }
    \or 

    % Ручная модификация
    \newcommand{\producttype}{Б.ХХХ}


    \newcommand{\producttechchar}
    {
        Поле 1  & 1 \\ \hline 
        Поле 2  & 2 \\ \hline %название не очень
        Поле 3  & 3 \\ \hline 
    }
    \newcommand{\productequip}
    {
        Поле 1  & 1 \\ \hline 
        Поле 2  & 2 \\ \hline %название не очень
        Поле 3  & 3 \\ \hline 
        Технический паспорт	& 1 \\ \hline
        Руководство по эксплуатации	& 1 \\ \hline
    }
    \fi
    %---Определение переменных зависящих от типа симулятора---

%%%%%%%%%%%%%%%%%%%%%%%%%%%%%%%%%%%%%%%%%%%%%%%%%%%%%%%%%%%%%%%%%%%%%%%%%%%%%%%%%%%%%%%%%%%%%%%%%%%%%%%%%%
%%%%%%%%%%%%%%%%%%%%%%%%%%%%%%%%%%%%%%%%%%%%%%%%%%%%%%%%%%%%%%%%%%%%%%%%%%%%%%%%%%%%%%%%%%%%%%%%%%%%%%%%%%
%%%%%%%%%%%%%%%%%%%%%%%%%%%%%%%%%%%%%%%%%%%%%%%%%%%%%%%%%%%%%%%%%%%%%%%%%%%%%%%%%%%%%%%%%%%%%%%%%%%%%%%%%%
%%%%%%%%%%%%%%%%%%%%%%%%%%%%%%%%%%%%%%%%%%%%%%%%%%%%%%%%%%%%%%%%%%%%%%%%%%%%%%%%%%%%%%%%%%%%%%%%%%%%%%%%%%
%%%%%%%%%%%%%%%%%%%%%%%%%%%%%%%%%%%%%%%%%%%%%%%%%%%%%%%%%%%%%%%%%%%%%%%%%%%%%%%%%%%%%%%%%%%%%%%%%%%%%%%%%%
%%%%%%%%%%%%%%%%%%%%%%%%%%%%%%%%%%%%%%%%%%%%%%%%%%%%%%%%%%%%%%%%%%%%%%%%%%%%%%%%%%%%%%%%%%%%%%%%%%%%%%%%%%

\else % определение переменных на английском языке

    %---Определение значения переменных не зависящих от типа симулятора---
    \newcommand{\productname}{Newborn Patient Simulator}

    %---Определение значения переменных не зависящих от типа симулятора---

    %---Определение переменных зависящих от типа симулятора---
    \ifcase \typesimulator
    
    % PS.N.HF
    \newcommand{\producttype}{PS.N.HF}

    %---Комплектация---
    \newcommand{\productequip}
    {
        Patient simulator & \\ \hline
        Simulator battery charger & \\ \hline

        Adapter & \\ \hline
        Adapter charger & \\ \hline
        4-electrode adapter cable & \\ \hline
        Adapter - ECG channel conversion cable & \\ \hline
        Adapter - Pacing channel conversion cable & \\ \hline
        Adapter - SpO2 channel conversion cable & \\ \hline
        Defibrillator pads & \\ \hline
        
        Blood pressure monitor with phonendoscope  & \\ \hline
        Drainage tube & \\ \hline
        
        Scalp vein infusion consumables & \\ \hline
        Hemothorax consumables & \\ \hline        
        
        Intraosseous infusion consumables & \\ \hline
        
        Clothing (baby cap) & \\ \hline
        Laptop & \\ \hline
        
        All-in-one PC & \\ \hline
        
        Wi-Fi router & \\ \hline
        
        Product Data Sheet & \\ \hline
        
        User Manual & \\ \hline

        %&\\ \hline
        %&\\ \hline
        %&\\ \hline 
    }

    \or

    %PS.N.ES
    \newcommand{\producttype}{PS.N.ES}

    %---Комплектация---
    \newcommand{\productequip}
    {
       
            
        
        &\\ \hline
        &\\ \hline
        &\\ \hline
        %&\\ \hline
        %&\\ \hline    
    }
    \or

    %PS.N.B
    \newcommand{\producttype}{PS.N.B}

    %---Комплектация---
    \newcommand{\productequip}
    {
        Поле 1  & 1 \\ \hline 
        Поле 2  & 2 \\ \hline %название не очень
        Поле 3  & 3 \\ \hline 
        &\\ \hline
        &\\ \hline
        &\\ \hline
        &\\ \hline
        &\\ \hline 
    }
    \or

    % PS.N.CPR
    \newcommand{\producttype}{PS.N.CPR}

    %---Комплектация---
    \newcommand{\productequip}
    {
             
        &\\ \hline 
    }
    \or 

    % Ручная модификация
    \newcommand{\producttype}{Б.ХХХ}


    \newcommand{\producttechchar}
    {
        Поле 1  & 1 \\ \hline 
        Поле 2  & 2 \\ \hline %название не очень
        Поле 3  & 3 \\ \hline 
    }
    \newcommand{\productequip}
    {
        Поле 1  & 1 \\ \hline 
        Поле 2  & 2 \\ \hline %название не очень
        Поле 3  & 3 \\ \hline 
        Технический паспорт	& 1 \\ \hline
        Руководство по эксплуатации	& 1 \\ \hline
    }  
    \fi
    %---Определение переменных зависящих от типа симулятора--- 

\fi











