%---Используемые переменные (справочный раздел)--- % Коммент этого сектора должен быть всегда, этот сектор для справки
% используются в операторах case
%\newcommand{\typesimulator}{0}   %тип симулятора (используется в операторе кейс) значения (0,1,2)=(Б,С,Г)
%\newcommand{\typemanufacturer}{0} % выбор компании

%обычные переменные
%\newcommand{\producttype}{Б.ЛПР} %тип продукта    
%\newcommand{\productname}{Виртуальный симулятор лапароскопии}
%\newcommand{\productmainview}{lapvision-standart} % внешний вид продукта    
%\newcommand{\productsertificate}{ce_sertificate_em} % сертификат продукта    
%\newcommand{\producttrademark}{LapVision} % торговое наименование симулятора
%\newcommand{\producttrademarktype}{Standart} % торговое наименование типа симулятора

%\newcommand{\productserialnumber}{/\_\_\_\_\_\_\_\_\_\_\_\_\_\_\_\_\_\_\_\_\_/}
%\newcommand{\productiondate}{2019}  %дата производства

%\newcommand{\productmadein}{Сделано в} % страна производства   
%\newcommand{\manufacturername}{ООО~«Эйдос-Медицина»} %название компании
%\newcommand{\address}{420107,  Россия, г.~Казань, ул.~Петербургская, д.50}
%\newcommand{\phone}{+7 (843) 227-40-63} 
%\newcommand{\email}{mail@oooeidos.ru}

%\newcommand{\producttechchar} % Технические характеристики симулятора
    %\newcommand{\producttechchardimension} % Габаритные размеры симулятора
    %\newcommand{\producttechcharweighs} % Вес симулятора 
%\newcommand{\productequip} % Комплектация симулятора
%---Используемые переменные (справочный раздел)---

%---Определение типа компании---  что-то одно всегда должно быть включено
\newcommand{\typemanufacturer}{0}   % ООО "Эйдос-Медицина" логотип "Медвижн"
%\newcommand{\typemanufacturer}{1}   % Другая компания и логотип
%---Определение типа компании---  

\ifdefined\typesimulator
\else
%---Определение типа симулятора---  что-то одно всегда должно быть включено
%\newcommand{\typesimulator}{0}   % Базовая модификация
%\newcommand{\typesimulator}{1}   % Стандартная модификация
\newcommand{\typesimulator}{2}   % Гибридная модификация
%\newcommand{\typesimulator}{3}     % Ручная модификация, если комплектацию и тип симулятора нужно вбить в ручную 
                                    %(редактировать ниже в операторе ifcase \typesimulator    на 3 позиции)
%---Определение типа симулятора---
\fi

%---Определение значения переменных не зависящих от типа симулятора---
\newcommand{\productname}{Виртуальный симулятор лапароскопии}
\newcommand{\producttrademark}{LapVision}
\newcommand{\productserialnumber}{\_\_\_\_\_\_\_\_\_\_\_\_\_\_\_\_\_\_\_\_\_}
\newcommand{\productiondate}{\_\_\_\_\_\_\_\_\_\_\_\_\_\_\_\_\_\_\_\_\_}
\newcommand{\productmadein}{Россия}
%---Определение значения переменных не зависящих от типа симулятора---

%---Определение значения переменных зависящих от типа компании--- 
\ifcase \typemanufacturer
%{ООО "Эйдос-Медицина}
\newcommand{\imagelogo}{medvision}
\newcommand{\productsertificate}{ce_sertificate_em}
\newcommand{\manufacturername}{ООО~«Эйдос-Медицина»}
\newcommand{\address}{420107,  Россия, г.~Казань, ул.~Петербургская, д.50}
\newcommand{\phone}{+7 (843) 227-40-63}
\newcommand{\email}{mail@oooeidos.ru}
\or 

%{Другая компания}
\newcommand{\imagelogo}{image-not-found}
\newcommand{\productsertificate}{image-not-found}
\newcommand{\manufacturername}{ООО~«ХХХХХХХХХ»}
\newcommand{\address}{ХХХХХХХХХХХХХХХХ}
\newcommand{\phone}{ХХХХХХХХХХХХХХ}
\newcommand{\email}{ХХХХХХХХХХХХ}
\fi
%---Определение значения переменных зависящих от типа компании--- 


%---Определение значения переменных по умолчанию--- 




%---Определение переменных зависящих от типа симулятора---
\ifcase \typesimulator
%{Базовая модификация} 
\newcommand{\producttype}{Б.ЛПР}
\newcommand{\productmainview}{lapvision-smart}
\newcommand{\producttrademarktype}{SMART}

\newcommand{\producttechchardimension}{550х650х650} % Габаритные размеры симулятора
\newcommand{\producttechcharweighs}{30} % Вес симулятора
%---Технические характеристики симулятора---
\newcommand{\producttechchar}
{
    Габаритные размеры (длина х ширина х высота) & \producttechchardimension мм \\\hline  
    Вес & \producttechcharweighs кг\\\hline
    Напряжение источника питания & 110 - 220 (\textpm 10 \%) В\\\hline
    Частота однофазной сети & 50/60 Гц\\\hline
    Количество розеток для подключения симулятора & 3 шт\\\hline
    Максимальная потребляемая мощность & 500 Вт\\\hline
    Площадь размещения & 4 м$^2$	\\\hline
    Рекомендуемый размер помещения & 3x3 м или больше\\\hline
    Условия эксплуатации & {\bf При использовании:} \newline\newline
    Температура воздуха в помещении: 10 – 25 \textdegree{}С; \newline\newline
    Относительная влажность воздуха: 30 – 75 \% \newline(без образования конденсата);\newline\newline
    Концентрация пыли в воздухе: \newline
    Максимальная разовая – 0,5 мг/м$^3$;\newline
    Среднесуточная – 0,15 мг/м$^3$;\newline
    \newline
    {\bf При хранении:}\newline\newline
    Температура воздуха в помещении: 10 - 40 \textdegree{}С; \newline\newline
    Относительная влажность воздуха 30 - 75\% \newline(без образования конденсата);\newline\newline
    Концентрация пыли в воздухе: \newline
    Максимальная разовая – 0,5 мг/м$^3$;\newline
    Среднесуточная – 0,15 мг/м$^3$.\\\hline
    Время непрерывной работы симулятора & 4 часа\\\hline
    Перерыв в работе симулятора не менее & 30 минут\\\hline 
}
%---Технические характеристики симулятора---
%---Комплектация---
\newcommand{\productequip}
{
    Блок сопряжения  &  \\ \hline 
    Порт камеры  &  \\ \hline %название не очень
    Порт инструмента (левый)  &  \\ \hline 
    Порт инструмента (правый)  &  \\ \hline 
    Беспроводной инструмент  &  \\ \hline
    Инструмент эндовидеокамеры  &  \\ \hline 
    Двухклавишная педаль  &  \\ \hline 
    ЖК-монитор  &  \\ \hline 
    Монитор &  \\ \hline % Какой? 
    Системный блок  &  \\ \hline
    Сетевой фильтр  &  \\ \hline                
    Технический паспорт	&  \\ \hline
    Руководство по эксплуатации	&  \\ \hline
    Руководство администратора &  \\ \hline %отсутствует в тз       
    &\\ \hline
    &\\ \hline
    &\\ \hline
    &\\ \hline
    &\\ \hline 
}

\or

%{Стандартная модификация}
\newcommand{\producttype}{C.ЛПР}
\newcommand{\productmainview}{lapvision-standart}
\newcommand{\producttrademarktype}{STANDARD}

\newcommand{\producttechchardimension}{550х1000х1800} % Габаритные размеры симулятора
\newcommand{\producttechcharweighs}{90} % Вес симулятора
%---Технические характеристики симулятора---
\newcommand{\producttechchar}
{
    Габаритные размеры (длина х ширина х высота) & \producttechchardimension мм \\\hline  
    Вес & \producttechcharweighs кг\\\hline
    Напряжение источника питания & 110 - 220 (\textpm 10 \%) В\\\hline
    Частота однофазной сети & 50/60 Гц\\\hline
    Количество розеток для подключения симулятора & 3 шт\\\hline
    Максимальная потребляемая мощность & 1000 Вт\\\hline
    Площадь размещения & 4 м$^2$	\\\hline
    Рекомендуемый размер помещения & 3x3 м или больше\\\hline
    Условия эксплуатации & {\bf При использовании:} \newline\newline
    Температура воздуха в помещении: 10 – 25 \textdegree{}С; \newline\newline
    Относительная влажность воздуха: 30 – 75 \% \newline(без образования конденсата);\newline\newline
    Концентрация пыли в воздухе: \newline
    Максимальная разовая – 0,5 мг/м$^3$;\newline
    Среднесуточная – 0,15 мг/м$^3$;\newline
    \newline
    {\bf При хранении:}\newline\newline
    Температура воздуха в помещении: 10 - 40 \textdegree{}С; \newline\newline
    Относительная влажность воздуха 30 - 75\% \newline(без образования конденсата);\newline\newline
    Концентрация пыли в воздухе: \newline
    Максимальная разовая – 0,5 мг/м$^3$;\newline
    Среднесуточная – 0,15 мг/м$^3$.\\\hline
    Время непрерывной работы симулятора & 4 часа\\\hline
    Перерыв в работе симулятора не менее & 30 минут\\\hline 
}
%---Технические характеристики симулятора---
%---Комплектация---
\newcommand{\productequip}
{
    Основа симулятора с лапароскопическими портами  &  \\ \hline 
    Подставка для инструментов  &  \\ \hline 
    Кронштейн для монитора  &  \\ \hline 
    Стойка для монитора  &  \\ \hline 
    Двухклавишная педаль  &  \\ \hline 
    Инструмент эндовидеокамеры  &  \\ \hline 
    Беспроводной инструмент  &  \\ \hline 
    Зарядный кабель USB  &  \\ \hline % USB кабель для зарядки беспроводных инструментов
    Сенсорный монитор &  \\ \hline 
    ЖК-монитор  &  \\ \hline 
    Системный блок (встроен в основу симулятора)  &  \\ \hline 
    USB кабель А-В  &  \\ \hline 
    HDMI кабель  &  \\ \hline 
    Блок питания монитора  &  \\ \hline 
    Сетевой шнур компьютер-розетка &  \\ \hline %Сетевой кабель подключения компьютера
    Сетевой фильтр  &  \\ \hline 
    Технический паспорт	&  \\ \hline
    Руководство по эксплуатации	&  \\ \hline
    Руководство администратора &  \\ \hline %отсутствует в тз   
    &\\ \hline
    %&\\ \hline
    %&\\ \hline    
}
\or

%{Гибридная модификация} 
\newcommand{\producttype}{Г.ЛПР}
\newcommand{\productmainview}{lapvision-hybrid}
\newcommand{\producttrademarktype}{HYBRID}

\newcommand{\producttechchardimension}{3000х2800х1800} % Габаритные размеры симулятора
\newcommand{\producttechcharweighs}{170} % Вес симулятора
%---Технические характеристики симулятора---
\newcommand{\producttechchar}
{
    Габаритные размеры (длина х ширина х высота) & \producttechchardimension мм \\\hline  
    Вес & \producttechcharweighs кг\\\hline
    Напряжение источника питания & 110 - 240 (\textpm 10 \%) В\\\hline
    Частота однофазной сети & 50/60 Гц\\\hline
    Количество розеток для подключения симулятора & 3 шт\\\hline
    Максимальная потребляемая мощность & 2000 Вт\\\hline
    Площадь размещения & 4 м$^2$	\\\hline
    Рекомендуемый размер помещения & 3x3 м или больше\\\hline
    Условия эксплуатации & {\bf При использовании:} \newline\newline
    Температура воздуха в помещении: 10 – 25 \textdegree{}С; \newline\newline
    Относительная влажность воздуха: 30 – 75 \% \newline(без образования конденсата);\newline\newline
    Концентрация пыли в воздухе: \newline
    Максимальная разовая – 0,5 мг/м$^3$;\newline
    Среднесуточная – 0,15 мг/м$^3$;\newline
    \newline
    {\bf При хранении:}\newline\newline
    Температура воздуха в помещении: 10 - 40 \textdegree{}С; \newline\newline
    Относительная влажность воздуха 30 - 75\% \newline(без образования конденсата);\newline\newline
    Концентрация пыли в воздухе: \newline
    Максимальная разовая – 0,5 мг/м$^3$;\newline
    Среднесуточная – 0,15 мг/м$^3$.\\\hline
    Время непрерывной работы симулятора & 4 часа\\\hline
    Перерыв в работе симулятора не менее & 30 минут\\\hline 
}
%---Технические характеристики симулятора---
%---Комплектация---
\newcommand{\productequip}
{
    Робот-пациент взрослый  &  \\ \hline
    Зарядное устройство робота-пациента &  \\ \hline
    
    Смартскоп &  \\ \hline
    Зарядное устройство Смартскопа &  \\ \hline
    
    Адаптер ЭКГ &  \\ \hline
    Зарядное устройство Адаптера ЭКГ &  \\ \hline
    Кабель Адаптера ЭКГ на 4 электрода &  \\ \hline
    Кабель переходник Адаптер ЭКГ - ДФБ &  \\ \hline
    Одноразовые электроды ЭКГ (наклейки) &  \\ \hline
    
    Адаптер ДФБ &  \\ \hline
    Зарядное устройство Адаптера ДФБ &  \\ \hline
    Накладки на электроды ДФБ &  \\ \hline
    
    Папка с RFID метками и шприцами &  \\ \hline
    Порт внутривенного вливания (в сборе) &  \\ \hline
    Внутрикостное вливание (в сборе) &  \\ \hline
    Ключ внутрикостного вливания &  \\ \hline
    
    Тонометр с доработкой &  \\ \hline
    Стетоскоп &  \\ \hline
    Пульсоксиметр &  \\ \hline
    
    Шарф (на шею) &  \\ \hline
    Шарф (для конечностей) &  \\ \hline
    Одежда: шорты &  \\ \hline    
    
    Лубрикант (флакон распылитель со смазкой) &  \\ \hline
    Расходники для коникотомии (клейкая лента) &  \\ \hline
    
    Имитатор операционной стойки &  \\ \hline
    Имитатор операционного стола &  \\ \hline
    Стойка для двух мониторов &  \\ \hline
    
    Блок сопряжения &  \\ \hline
    Имитатор блока инсуффлятора &  \\ \hline
    Имитатор блока коагулятора &  \\ \hline    
    Имитатор блока аспиратора-ирригатора &  \\ \hline
    Имитатор блока эндовидеокамеры &  \\ \hline
    Имитатор блока осветителя &  \\ \hline
    
    Педаль двухклавишная &  \\ \hline
    Педаль одноклавишная &  \\ \hline
    
    Порт камеры &  \\ \hline
    Порт инструмента (левый) &  \\ \hline
    Порт инструмента (правый) &  \\ \hline
    Имитаторы лапароскопических инструментов &  \\ \hline
    Имитатор эндовидеокамеры &  \\ \hline
    
    Системный блок &  \\ \hline
    Ноутбук &  \\ \hline
    Моноблок &  \\ \hline
    Монитор сенсорный &  \\ \hline
    Монитор &  \\ \hline
    Wi-Fi роутер  &  \\ \hline
    
    Web-камера &  \\ \hline
    Наушники с микрофоном &  \\ \hline
    
    Сетевой фильтр &  \\ \hline    
    
    Технический паспорт &  \\ \hline
    Руководство по эксплуатации &  \\ \hline    
    Руководство администратора &  \\ \hline

    &\\ \hline
    &\\ \hline
    &\\ \hline
    &\\ \hline
    &\\ \hline 
}
\or 

% Ручная модификация
\newcommand{\producttype}{Б.ХХХ}
\newcommand{\productmainview}{image-not-found}
\newcommand{\producttrademarktype}{NoName}

\newcommand{\producttechchardimension}{XXXхXXхXXXXX} % Габаритные размеры симулятора
\newcommand{\producttechcharweighs}{XXX} % Вес симулятора

\newcommand{\producttechchar}
{
    Поле 1  & 1 \\ \hline 
    Поле 2  & 2 \\ \hline %название не очень
    Поле 3  & 3 \\ \hline 
}
\newcommand{\productequip}
{
    Поле 1  & 1 \\ \hline 
    Поле 2  & 2 \\ \hline %название не очень
    Поле 3  & 3 \\ \hline 
    Технический паспорт	& 1 \\ \hline
    Руководство по эксплуатации	& 1 \\ \hline
}
\fi
%---Определение переменных зависящих от типа симулятора---



