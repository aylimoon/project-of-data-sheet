\newif\ifIsEnsim\IsEnsimtrue%взаимоисключающие с vision и med
\newif\ifIsMed\IsMedfalse  %взаимоисключающие с vision и ensim
\newif\ifIsVision\IsVisionfalse %взаимоисключающие с med и ensim


\newif\ifIsNO\IsNOfalse %когда нужно лишнее убрать

\newif\ifIsProductdata\IsProductdatafalse % Использование данных о продукте по умолчанию (если не сгенерирован файл о данных продукта из задачи)

\newif\ifIsLngRus\IsLngRustrue % Флаг использования рускоязычных секций из файлов
%\IsLngRusfalse % отключение языковой секции (рус) при комиляции   % должно быть закомичено при компиляцией bat файлом
\newif\ifIsLngEng\IsLngEngtrue % Флаг использования англоязычных секций из файлов
%\IsLngEngfalse  % отключение языковой секции (англ) при комиляции % должно быть закомичено при компиляцией bat файлом

% флаги и настройки не удалять, просто комментировать что нужно отключить
\documentclass[
10pt, 
twoside  % тут можно отключить двустороннюю печать
]{extarticle}




 \usepackage{ifthen} %для макрофлагов
 \usepackage{longtable} % для таблиц на нескольких страницах
\usepackage{graphicx} % Подключаем пакет работы с графикой
 \usepackage[usenames]{color} % этот и следующий пакеты для работы с цветом
 \usepackage{fancyhdr} % колонтитулы
 \usepackage{indentfirst} % отступ для первого абзаца главы или параграфа
% \usepackage{currfile}
 \usepackage{colortbl} %раскрашивать колонки в таблица
 \usepackage{multirow} %картинки в ряд (для floatrow)
 \usepackage{titlesec}
 \usepackage{caption}
 \usepackage{subcaption}
 \usepackage{floatrow} % управление плавающими объектами (картинки, таблицы)
 \usepackage{calc} % изменение размера для подписи, совместно с floatrow
 \newcommand{\sectionbreak}{\clearpage}
\usepackage{color}
\usepackage{textcomp} % специальные символы
\usepackage{afterpage} % команда после страницы
\usepackage{pagecolor}% цвет страницы

%--------Формат подписей к рисункам и таблицам--------
\usepackage{caption}
\DeclareCaptionLabelSeparator{dot}{. }
\captionsetup{justification=centering, labelsep=dot}
%--------Формат подписей к рисункам и таблицам--------

%--------ШРИФТ--------
\fontsize{10pt}{14pt}\selectfont % установка размера шрифта для всего документа
\usepackage{setspace}\onehalfspacing % интервал полуторный
%\renewcommand{\rmdefault}{ptm} % использовать Times New Roman
% \renewcommand{\familydefault}{\sfdefault} % Latin Modern Sans Serif шрифт  \ не включать корежит шрифт Arial на английском

\frenchspacing %обычные пробелы после знаков препинания, а не увеличенные дефолтные как в англо-саксонской типографике
%--------ШРИФТ--------


%----------------SETTINGS----------------
\def\pathtemplate{multilang} % папка с файлами шаблона проекта
\def\pathvariables{multilang/variables} % папка с файлами определяющими переменные на все симуляторы

\usepackage{polyglossia}   %% загружает пакет многоязыковой вёрстки

% Язык менять только тут, раскомментировать нужный, остальный закомментировать

\ifdefined\lngflag % \lngflag выбирает язык загрузки и определяется через bat файл, значение 0 - русский язык, 1- английский
    \ifcase\lngflag      
        \setdefaultlanguage{russian} % определение языка
        \setotherlanguage{english}
        \IsLngEngfalse % отключение англ информации 
    \else
        \setdefaultlanguage{english} % определение языка  
        \setotherlanguage{russian}      
        \IsLngRusfalse % отключение рус информации
    \fi 
\else

\setdefaultlanguage{russian} % установка языка
%\setdefaultlanguage{english} % установка языка%\def\lng{russian} 
%\def\lng{english} % Язык (при отладке в VSC) менять только тут, раскомментировать нужный, остальный закомментировать
\fi

% \usepackage{xltxtra}

% Основной шрифт документа
\defaultfontfeatures{Ligatures=TeX,Mapping=tex-text}
\setmainfont{Arial}
%\setmainfont{Times New Roman}
%\newfontfamily\cyrillicfont{Arial}
\newfontfamily{\cyrillicfontsf}{Arial}
%\newfontfamily{\cyrillicfontsf}{Times New Roman}
%----------------SETTINGS----------------





%--------ПОЛЯ ЛИСТА--------
\usepackage[
a5paper,
left=2.5cm,
right=1.5cm,
top=1.5cm,
bottom=1.5cm,
headheight=1.62cm,
headsep=0.5cm,
footskip=1.5cm,
includehead, includefoot,
%showframe % УДОБНАЯ ВЕЩЬ, но не забыть закоментить перед релизом
]{geometry}
%--------ПОЛЯ ЛИСТА--------

%--------Constants--------
% \ifIsEnsim
% \definecolor{clColontitulLine}{rgb}{0, 0.44, 0.72}
%\definecolor{clColontitulText}{rgb}{0, 0.44, 0.72}
\definecolor{clColontitulText}{RGB}{0, 65, 153}

% \else
%\definecolor{clColontitulLine}{rgb}{0.14, 0.14, 0.14}
\definecolor{clColontitulLine}{RGB}{0, 65, 153}
\definecolor{PageColorGround}{RGB}{0, 65, 153}

\definecolor{TradeMarkNameOneColor}{RGB}{0, 65, 153}  % цвет букв на титульном листе % темно синий 
%\definecolor{TradeMarkNameOneColor}{RGB}{255,255,255} % белый
\definecolor{TradeMarkNameTwoColor}{RGB}{240, 124, 0} % оранжевый

% \fi
% \def\uColontitulLineW{2mm}
\def\uColontitulLineW{1mm}
%--------Constants--------

%--------Bookmarks(ссылки/оглавление)--------
\usepackage[
colorlinks, 
urlcolor=clColontitulLine, 
linkcolor=clColontitulLine, 
bookmarksopen=true,
bookmarksnumbered=true,
pagebackref=true,
breaklinks=true,
]{hyperref}
%--------отключение переносов в тексте--------
\tolerance=1
\emergencystretch=\maxdimen
\hyphenpenalty=10000
\hbadness=10000
%--------отключение переносов в тексте--------